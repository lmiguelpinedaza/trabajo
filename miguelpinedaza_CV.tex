%% start of file `template.tex'.
%% Copyright 2006-2015 Xavier Danaux (xdanaux@gmail.com).
%
% This work may be distributed and/or modified under the
% conditions of the LaTeX Project Public License version 1.3c,
% available at http://www.latex-project.org/lppl/.


\documentclass[11pt,a4paper,sans]{moderncv}        % possible options include font size ('10pt', '11pt' and '12pt'), paper size ('a4paper', 'letterpaper', 'a5paper', 'legalpaper', 'executivepaper' and 'landscape') and font family ('sans' and 'roman')

% moderncv themes
\moderncvstyle{casual}                             % style options are 'casual' (default), 'classic', 'banking', 'oldstyle' and 'fancy'
\moderncvcolor{grey}                               % color options 'black', 'blue' (default), 'burgundy', 'green', 'grey', 'orange', 'purple' and 'red'
%\renewcommand{\familydefault}{\sfdefault}         % to set the default font; use '\sfdefault' for the default sans serif font, '\rmdefault' for the default roman one, or any tex font name
%\nopagenumbers{}                                  % uncomment to suppress automatic page numbering for CVs longer than one page

% character encoding
\usepackage[utf8]{inputenc}                       % if you are not using xelatex ou lualatex, replace by the encoding you are using
%\usepackage{CJKutf8}                              % if you need to use CJK to typeset your resume in Chinese, Japanese or Korean

% adjust the page margins
\usepackage[scale=0.75]{geometry}
%\setlength{\hintscolumnwidth}{3cm}                % if you want to change the width of the column with the dates
%\setlength{\makecvtitlenamewidth}{10cm}           % for the 'classic' style, if you want to force the width allocated to your name and avoid line breaks. be careful though, the length is normally calculated to avoid any overlap with your personal info; use this at your own typographical risks...

%comentarios
\usepackage{comment}

% personal data
\name{Luis Miguel}{Pineda Daza}
\title{Licenciatura en Matemáticas Aplicadas y Computación}                               % optional, remove / comment the line if not wanted
%(actualmente)

\address{Barranca Chica \#88 int 17}{Naucalpan}{Estado de México}% optional, remove / comment the line if not wanted; the "postcode city" and "country" arguments can be omitted or provided empty
\phone[mobile]{5525017741}                   % optional, remove / comment the line if not wanted; the optional "type" of the phone can be "mobile" (default), "fixed" or "fax"
%\phone[fixed]{+2~(345)~678~901}
%\phone[fax]{+3~(456)~789~012}
\email{miguelpinedaza@gmail.com}                               % optional, remove / comment the line if not wanted
%\homepage{satakshi.in/purnendu}                         % optional, remove / comment the line if not wanted
%\social[linkedin]{purnenduk90}                        % optional, remove / comment the line if not wanted
%\social[twitter]{purnenduk90}                             % optional, remove / comment the line if not wanted
%\social[github]{PurnenduK90}                              % optional, remove / comment the line if not wanted
%\extrainfo{additional information}                 % optional, remove / comment the line if not wanted
\photo[64pt][0.4pt]{C:/Users/Acer/Documents/trabajo/foto-CV}                       % optional, remove / comment the line if not wanted; '64pt' is the height the C:/Users/Acer/Documents/trabajo/template-overleaf/picture must be resized to, 0.4pt is the thickness of the frame around it (put it to 0pt for no frame) and 'C:/Users/Acer/Documents/trabajo/template-overleaf/picture' is the name of the C:/Users/Acer/Documents/trabajo/template-overleaf/picture file
\quote{Si algo es escaso de forma natural, \newline es más probable que tenga valor.}                                 % optional, remove / comment the line if not wanted

% bibliography adjustements (only useful if you make citations in your resume, or print a list of publications using BibTeX)
%   to show numerical labels in the bibliography (default is to show no labels)
\makeatletter\renewcommand*{\bibliographyitemlabel}{\@biblabel{\arabic{enumiv}}}\makeatother
%   to redefine the bibliography heading string ("Publications")
%\renewcommand{\refname}{Articles}

% bibliography with mutiple entries
%\usepackage{multibib}
%\newcites{book,misc}{{Books},{Others}}
%----------------------------------------------------------------------------------
%            content
%----------------------------------------------------------------------------------
\begin{document}
%\begin{CJK*}{UTF8}{gbsn}                          % to typeset your resume in Chinese using CJK
%-----       resume       ---------------------------------------------------------
\makecvtitle

%\section{Educación}
%\cventry{}{Matemáticas Aplicadas y Computación}{Facultad de Estudios Superiores - Acatlán}{}{}{}  % arguments 3 to 6 can be left empty
%\cventry{2009--2013}{BE}{G H R C E}{Nagpur}{\textit{73.09\%}}{Electrical Engineering, Electronics and Power}

%\section{Master thesis}
\section{Objetivos}
%\cvitem{title}{\emph{Design and characterization of discrete analog front-end for resistive plate chamber (RPC) detector}}
%\cvitem{supervisors}{Dr. Aniruddhan Sankaran, Dr. Anil Prabhakar}
%\cvitem{description}{Resistive Plate Chamber (RPC) detector gives nanosecond electrical pulses with a few millivolts of amplitude at 50 Ohm termination impedance. For accurate timing data abstraction using a precision TDC, it is required to have a very accurate front-end with lowest possible time-walk and jitter in the output signal. Thesis emphasis on design of high gain broadband amplifier, and to characterize CFD with varying delays to get best possible slope on both rising edge and falling edge of signal for precision time measurement. A 0.35 ns improvement over 1.55ns of leading edge discriminator was obtained by use of ARCD topology and ac coupling for fast return to zero.}
\cvitem{}{Obtener un puesto en el área especifica aplicando mis conocimientos adquiridos y contribuir astutamente en el beneficio de la empresa.}

\begin{comment}
    Perfil del profesionista

    El licenciado en Matemáticas Aplicadas y Computación es un profesionista capaz de utilizar las matemáticas y la computación de manera creativa para formular, analizar, diseñar, construir y automatizar soluciones a problemas reales. Durante su desempeño profesional ejercerá sus habilidades para actuar en equipos y adaptar métodos abstractos a la solución de problemas de orden práctico, así como a la modelación matemática y computacional de situaciones reales, con un pensamiento crítico, creativo e innovador de naturaleza inter y multidisciplinaria.

    Estará capacitado para desempeñar actividades como:
    Identificar problemas y proponer soluciones. Proponer constructos matemáticos-computacionales. Participar en equipos de investigación aplicada y documental en tecnologías de información, comunicación y desarrollo de software y hardware, para apoyar los procesos y servicios de una organización. Ofrecer consultoría en áreas físico-matemáticas y económico-administrativas, en inteligencia artificial, en tecnologías de la información, sistemas y programas de última generación. Atender las necesidades empresariales a través de la capacitación o actualización académicas. Desarrollar y manejar software: de sistema, de aplicación para resolver necesidades específicas de negocio, científico, empotrado, de línea de producto, así como aplicaciones de inteligencia artificial basadas en la Web y dispositivos móviles. Desempeñar la docencia en niveles de pregrado.
    Objetivo

    Desarrollar en el alumno la capacidad de aplicar creativamente las matemáticas y técnicas computacionales para analizar, evaluar y resolver problemas por medio de modelos en diversas áreas de conocimiento.
    Características y habilidades recomendables del estudiante

    El estudiante de Matemáticas Aplicadas y Computación debe poseer los conocimientos necesarios del área físico-matemática, contar con facilidad y razonamiento lógico, capacidad de concentración, de análisis y síntesis; tener una gran creatividad y curiosidad científica, así como de disciplina y constancia en el estudio y habilidad para el trabajo en equipo.
\end{comment}

\section{Conocimiento}
%\cvitem{}{Experiencia en la plataforma de eCommerce MAGENTO, realizando customizaciones de archivos  css / xml / phtml / js así mismo la creación de módulos y extensiones.}
\cvitem{}{Experiencia en la plataforma de eCommerce MAGENTO, realizando customizaciones de archivos  css / xml / phtml / js así mismo la creación de módulos y extensiones. Se utilizó la programación orientada a objetos y el backwards-compatible "LESS" para estilos CSS.}
%\cvitem{}{Manejo de shell en los sistemas operativos Unix y Windows.}
\cvitem{}{Manejo de shell en los sistemas operativos Unix y Windows. Se trabaja con el editor de texto ``vim'' y LaTex para creación de documentos.}% como procesador de texto.}
%\cvitem{}{Conocimiento de lenguajes de programación: C, C++,  SQL, Python, Java, JavaScript, Html5, PHP y CSS, y manejo de Office (excel avanzado). }
\cvitem{}{Conocimiento de lenguajes de programación: C, C++,  SQL, Python, Java, R (estadística), JavaScript, Html5, PHP y CSS; y manejo de Office (excel avanzado).}
\cvitem{}{Mantenimiento de paginas web (Performance, Accessibility, Best Practices, SEO y PWA).}
\cvitem{}{Uso de IDE: Netbeans, Notepad++, Sublime text y Visual Studio. Administración y configuración en plataformas de google (Google Console, Google Analytics, Google Trends, Google Ads, GTM, Google Optimize). Manejo de control de versiones con github y gitlab.}
%\cvitem{}{Realice mi servicio social en el Programa de Apoyo a Proyectos para la Innovación y Mejoramiento de la Enseñanza (PAPIME), realizando un Taller De Mantenimiento y Soporte T\'ecnico %con el tema de Instalaci\'on y Configuraci\'on de un Sistema Operativo }
%con el Dr. Eduardo Eloy Loza Pacheco.}
\cvitem{}{Realicé mi servicio social en el Programa de Apoyo a Proyectos para la Innovación y Mejoramiento de la Enseñanza (PAPIME), realizando un Taller De Mantenimiento y Soporte T\'ecnico instalando en un mismo equipo diferentes sistemas operativos.%con el tema de Instalaci\'on y Configuraci\'on de un Sistema Operativo }
con el Dr. Eduardo Eloy Loza Pacheco. }
\newpage
\section{Experiencia laboral}
%\subsection{Regular}

\cventry{2020-8 \newline ...}{Instituto Mexicano de Desarrollo de Software}{Desarrollador Magento}{}{}{
    %Achievements:%
    \begin{itemize}%
	\item Mantenimiento de diferentes paginas web.
	    %\item Update of RPC-DAQ schematic for switched power supply.
	    %\item Test methodology development for RPC Test-jig, Pin-Assignment for FPGA, and schematic design.
	    %\item Design of LC ladder based sub-nanosecond stepped electronically controlled delay circuit.
    \end{itemize}}

\cventry{2019-11\newline 2020-6}{doto.com.mx}{Desarrollador Magento}{}{}{
    %Achievements:%
    \begin{itemize}%
	\item Mantenimiento de la pagina web doto.com.mx desarrollando back-end y front-end.
	    %\item Update of RPC-DAQ schematic for switched power supply.
	    %\item Test methodology development for RPC Test-jig, Pin-Assignment for FPGA, and schematic design.
	    %\item Design of LC ladder based sub-nanosecond stepped electronically controlled delay circuit.
    \end{itemize}}

\cventry{2019-8 \newline 2019-9}{Promociones y Display Marketing}{Sistemas y Soporte}{}{}{
    %Achievements:%
    \begin{itemize}%
	\item Manejo de base de datos, generando reportes y altas de eventos, asignación de usuarios y contraseñas.
	    %\item Update of RPC-DAQ schematic for switched power supply.
	    %\item Test methodology development for RPC Test-jig, Pin-Assignment for FPGA, and schematic design.
	    %\item Design of LC ladder based sub-nanosecond stepped electronically controlled delay circuit.
    \end{itemize}}

\cventry{2016-6\newline 2016-8 2017-6\newline 2017-8}{McDonald's, Naucalapan, Méx}{Empleado General}{}{}{
    %Achievements:%
    \begin{itemize}%
	\item Conocimiento en cocina y atención al cliente.
	    %\item Update of RPC-DAQ schematic for switched power supply.
	    %\item Test methodology development for RPC Test-jig, Pin-Assignment for FPGA, and schematic design.
	    %\item Design of LC ladder based sub-nanosecond stepped electronically controlled delay circuit.
    \end{itemize}}

\cventry{2016-11 \newline 2017-1}{Empacador de pan, MEGA San Mateo.}{Empleado General}{}{}{
    %Achievements:%
    \begin{itemize}%
	\item Elaboración del producto y atención al cliente
	    %\item Update of RPC-DAQ schematic for switched power supply.
	    %\item Test methodology development for RPC Test-jig, Pin-Assignment for FPGA, and schematic design.
	    %\item Design of LC ladder based sub-nanosecond stepped electronically controlled delay circuit.
    \end{itemize}}

\cventry{2015-2 \newline 2015-9}{Burger King M\'exico - Naucalpan, M\'ex}{Líder de producción}{}{}{
    %Achievements:%
    \begin{itemize}%
	\item Manejo de tiempos y temperaturas de la producción del producto así como la evaluación de empleados.
	    %\item Update of RPC-DAQ schematic for switched power supply.
	    %\item Test methodology development for RPC Test-jig, Pin-Assignment for FPGA, and schematic design.
	    %\item Design of LC ladder based sub-nanosecond stepped electronically controlled delay circuit.
    \end{itemize}}

\cventry{...}{Carpintería}{Carpintero}{}{}{
    %Achievements:%
    \begin{itemize}%
	\item En horarios no laborales apoyo al negocio familiar con trabajos de carpintería.
	    %\item Update of RPC-DAQ schematic for switched power supply.
	    %\item Test methodology development for RPC Test-jig, Pin-Assignment for FPGA, and schematic design.
	    %\item Design of LC ladder based sub-nanosecond stepped electronically controlled delay circuit.
    \end{itemize}}

\begin{comment}
    \cventry{2013--2014}{Junior Research Fellow}{University of Delhi}{Dept. of Physics and Astrophysics}{Delhi}{RPC Electrode characterization for INO Project\newline{}
    Achievements:
    \begin{itemize}
	\item Programming CAEN VME data acquisition system.
	\item Assembly of Plastic scintillator with photo-multiplier tube and characterization.
	\item Assembly and characterization of multiple glass and bakelite RPC.
    \end{itemize}}


    \subsection{Vocational}
    \cventry{Nov. 2011}{Intern}{Patratu Thermal Power Station}{Pataratu}{Jharkhand}{Transformers, switching stations}
    \cventry{Nov. 2010}{Intern}{Bharti Airtel Limited}{Patna}{Bihar}{Power management at tower site, centralized failure monitoring}

    \section{Languages}
    \cvitemwithcomment{C}{Intermediate}{2010 - Present, GCC, C99}
    \cvitemwithcomment{Python}{Intermediate}{2014 - Present, Spyder IDE}
    \cvitemwithcomment{Verilog}{Intermediate}{2015 - Present, DSP Architecture optimization, VLSI Lab}
    \cvitemwithcomment{C++}{Basic}{NIIT certification 2010, Visual CPP, QT, G++, C++11}
    \cvitemwithcomment{HTML,PHP}{Basic}{2015 - Present (satakshi.in)}

    \section{Computer skills}
    \cvdoubleitem{EDA}{Allegro, Eagle, KiCad}{Application}{MATLAB}
    \cvdoubleitem{Device Simulation}{Virtuoso-IC, Vivado, Quartus}{Circuit Simulation}{Spice Opus, NGSpice, LTSpice}
    \cvdoubleitem{CAD}{Wings3D, Sketch-up}{Documentation}{MS Office, Libre Office, tex}

    \section{Interests}
    \cvitem{Blogging}{blog.satakshi.in}
    \cvitem{Circuit design}{EEZ-PSU hardware issue debugging (github)}
    \cvitem{Coding}{Active on Hacker Rank (Algorithm challenge)}

    \section{Co Curricular}
    \cventry{Workshop}{MATLAB based Image Processing}{by MagicMan Technologies}{Mumbai}{06-2012}{Design of MATLAB based line/object/gesture follower robot}

    \section{Extra Curricular}
    \cvlistitem{Hovercraft design competition finalist at kshitij-2011 (IIT Kharagpur).}
    \cvlistitem{Co-ordinator at EPICS-2011, G H R C E, Nagpur.}
    \cvlistitem{Represented R. K. College, Madhubani in inter college table-tennis tournament (L. N. M. University,
    Darbhanga, Bihar) 2006.}

    \section{References}
    \begin{cvcolumns}
    \cvcolumn{IITM}{\begin{itemize}\item Dr. Anil P.\item Dr. Aniruddhan S.\item Dr. P.K.Behera\end{itemize}}
	    \cvcolumn{DU}{\begin{itemize}\item Dr. Md. Naimuddin, \item Dr. Ashok K.\end{itemize}}
			\cvcolumn[0.4]{All the rest \& some more}{\textit{Dr. S.B.Bodke (GHRCE)}, and \textit{Dr. Satyanarayna B. (TIFR)} }
    \end{cvcolumns}
\end{comment}

% Publications from a BibTeX file without multibib
%  for numerical labels: \renewcommand{\bibliographyitemlabel}{\@biblabel{\arabic{enumiv}}}% CONSIDER MERGING WITH PREAMBLE PART
%  to redefine the heading string ("Publications"): \renewcommand{\refname}{Articles}
\nocite{*}
\bibliographystyle{unsrt}
\bibliography{publications}                      % 'publications' is the name of a BibTeX file

% Publications from a BibTeX file using the multibib package
%\section{Publications}
%\nocitebook{book1,book2}
%\bibliographystylebook{plain}
%\bibliographybook{publications}                   % 'publications' is the name of a BibTeX file
%\nocitemisc{misc1,misc2,misc3}
%\bibliographystylemisc{plain}
%\bibliographymisc{publications}                   % 'publications' is the name of a BibTeX file

%\clearpage
%-----       letter       ---------------------------------------------------------
% recipient data
% \recipient{Company Recruitment team}{Company, Inc.\\123 somestreet\\some city}
% \date{January 01, 1984}
% \opening{Dear Sir or Madam,}
% \closing{Yours faithfully,}
% \enclosure[Attached]{curriculum vit\ae{}}          % use an optional argument to use a string other than "Enclosure", or redefine \enclname
% \makelettertitle

% Lorem ipsum dolor sit amet, consectetur adipiscing elit. Duis ullamcorper neque sit amet lectus facilisis sed luctus nisl iaculis. Vivamus at neque arcu, sed tempor quam. Curabitur pharetra tincidunt tincidunt. Morbi volutpat feugiat mauris, quis tempor neque vehicula volutpat. Duis tristique justo vel massa fermentum accumsan. Mauris ante elit, feugiat vestibulum tempor eget, eleifend ac ipsum. Donec scelerisque lobortis ipsum eu vestibulum. Pellentesque vel massa at felis accumsan rhoncus.

% Suspendisse commodo, massa eu congue tincidunt, elit mauris pellentesque orci, cursus tempor odio nisl euismod augue. Aliquam adipiscing nibh ut odio sodales et pulvinar tortor laoreet. Mauris a accumsan ligula. Class aptent taciti sociosqu ad litora torquent per conubia nostra, per inceptos himenaeos. Suspendisse vulputate sem vehicula ipsum varius nec tempus dui dapibus. Phasellus et est urna, ut auctor erat. Sed tincidunt odio id odio aliquam mattis. Donec sapien nulla, feugiat eget adipiscing sit amet, lacinia ut dolor. Phasellus tincidunt, leo a fringilla consectetur, felis diam aliquam urna, vitae aliquet lectus orci nec velit. Vivamus dapibus varius blandit.

% Duis sit amet magna ante, at sodales diam. Aenean consectetur porta risus et sagittis. Ut interdum, enim varius pellentesque tincidunt, magna libero sodales tortor, ut fermentum nunc metus a ante. Vivamus odio leo, tincidunt eu luctus ut, sollicitudin sit amet metus. Nunc sed orci lectus. Ut sodales magna sed velit volutpat sit amet pulvinar diam venenatis.

% Albert Einstein discovered that $e=mc^2$ in 1905.

%\[ e=\lim_{n \to \infty} \left(1+\frac{1}{n}\right)^n \]

% \makeletterclosing

% %\clearpage\end{CJK*}                              % if you are typesetting your resume in Chinese using CJK; the \clearpage is required for fancyhdr to work correctly with CJK, though it kills the page numbering by making \lastpage undefined
\end{document}


%% end of file `template.tex'.

